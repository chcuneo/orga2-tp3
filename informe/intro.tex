\section{Introducción}

El objetivo del presente trabajo practico es aprender y aplicar diferentes conceptos de \textit{System Programming}. A partir de una implementación de un boot-sector, se programo un pequeño kernel con los diferentes mecanismos de protección y ejecución concurrente de tareas para luego poder ejecutar un juego con hasta 16 tareas concurrentes a nivel de usuario.

\subsection{Inicialización}

Al prender la computadora, comienza la inicializacion del \texttt{POST} (Power-On Self-Test), un programa de diagnostico de hardware que verifica que todos los dispositivos se han inicializado de manera correcta. Una vez terminado el \texttt{POST}, el \texttt{BIOS} se encarga de identificar el primer dispositivo de boooteo, ya sea un CD, un disco rígido o un diskette. En este trabajo, inicializaremos el sistema a partir de un diskette.

El \texttt{BIOS} (Basic Input-Output System) copia de memoria \texttt{RAM} los primeros 512 bytes del sector a partir de la direccion \addr{0x7c00} de un diskette. Esto se copia comenzando en la direccion \addr{0x1200} y luego se ejecuta el boot-sector a partir de allí. El boot-sector encuentra en el floppy el archivo \file{kernel.bin}, y luego lo copia en memoria a partir de la direccion \addr{0x1200}, ejecutando a partir de la misma.

