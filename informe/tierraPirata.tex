\section{Tierra Pirata}

\subsection{Memory Management Functions}
Para facilitar el manejo del armado de estructuras para la paginación, se crearon las siguientes funciones en C. Antes de explicar que hace cada función, un comentario. Cada directorio de paginas tiene 1024 entradas de descriptores de 4 bytes. Lo mismo sucede con los directorios de paginas, que también tienen 1024 entradas con descriptores de 4 bytes. El procesador, al buscar estas estructuras en memoria RAM, requiere que las mismas estén alineadas a 4kb, dado que es el tamaño de pagina que carga en memoria cache.

\begin{enumerate}
\item \fun{create\_page\_table(uint directoryBase, uint directoryEntry, uint physicalAddress, uchar readWrite, uchar userSupervisor)}: Asigna una \texttt{page\_table} a una tabla de directorios con los atributos pasados por parametro. Al final de la función, se limpia la memoria cache para garantizar que cuando el procesador busca esta pagina, la misma se encuentra actualizada.

\item \fun{delete\_page\_table(uint directoryBase, uint directoryEntry)}: Borra una tabla de paginas de un directorio de paginas. Esto lo hace simplemente setteando el bit P (present) en cada pagina en 0.

\item \fun{create\_page(uint directoryBase, uint directoryEntry, uint tableEntry, uint physicalAddress, \\ uchar readWrite, uchar userSupervisor)}: Crea una pagina en la tabla de paginas de algún directorio.

\item \fun{delete\_page(uint directoryBase, uint directoryEntry, uint tableEntry)}: Borra una pagina en la tabla de paginas de algún directorio. Esto lo hace setteando el bit P en 0.

\item \fun{mmap(uint virtualAddress, uint physicalAddress, uint directoryBase, uchar readWrite, \\ uchar userSupervisor)}: Mappea una dirección virtual en una direccion fisica. Para esto, primero se busca la tabla de paginas y la pagina correspondiente a la dirección virtual. Luego se le asigna a esa pagina la dirección física. Esto se hace de la siguiente forma:
	\begin{enumerate}
	\item A partir de la dirección virtual, se busca la entrada de directorio correspondiente a la misma. Esto se hace dividiendo el virtualAdress por el tamaño direccionable por cada page\_directory.$virtualAdress / 1024*4kb$. Esto es equivalente a $virtualAdress >> 22$.
	\item Buscamos el indice en la entrada de paginas. Esto se calcula dividiendo por el tamaño de pagina e ignorando los bits correspondientes a la entrada de directorio $virtualAdress / 4kb$ $\&$ $0x3FF$, que es equivalente a $virtualAdress >> 12$ $\&$ $0x3FF$.
	\end{enumerate}

\item \fun{munmap(uint directoryBase, uint virtualAddress)}: Desmappea la pagina correspondiente a una dirección virtual. Calcula todos los indices necesarios de la misma manera que \fun{mmap}

\item \fun{mmu\_inicializar\_dir\_kernel()}: Inicializa el directorio del kernel. Para ello, hacemos memory mapping sobre el kernel y le asignamos un area libre, todo desde \addr{0x00000000} a \addr{0x003FFFF}.

\item \fun{mmu\_inicializar\_dir\_pirata(uint directoryBase, uint pirateCodeBaseSrc, uint pirateCodeBaseDst)}: Esta función inicializa el directorio de un pirata. Al igual que el Kernel, hacemos memory mapping, aunque en modo user y en read only. A su vez, mappeamos la pagina donde vamos a poner el codigo del pirata, y copiamos el código del pirata que se encuentra en el Kernel en esta pagina.

\item \fun{mmu\_move\_codepage(uint src, uint dst, pirata\_t *p)}: Mueve la pagina de codigo del pirata desde $src$ a $dst$.

No implementamos la funcion \texttt{mmu\_inicializar} dado que todo el trabajo lo hace \texttt{mmu\_inicializar\_dir\_kernel}.

\end{enumerate}

\subsection{Scheduler}

\subsection{Estructuras}

\subsection{Funciones Auxiliares}

\subsection{Piratas}

\subsubsection{Explorador}

\subsubsection{Minero}
